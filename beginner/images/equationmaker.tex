\documentclass{beamer}
\usepackage{framed}
\usepackage{enumerate}
\usepackage{amsmath}
\usepackage{multicol}

\begin{document}
\begin{frame}
	\large
Given \texttt{x = [3 1 5 7 9 2 6]}, explain what the following commands do by
summarizing the net result of the command.

\begin{multicols}{2}
\begin{enumerate}[(i)]
	\item  \texttt{x(3)}
	\item \texttt{x(1:5)}
	\item \texttt{x(1:end)}
	\item \texttt{x(1:end-1)}
	\item \texttt{x}\texttt{(6:-2:1)}
	\item \texttt{x([1 6 2 1 1])}
	\item \texttt{sum(x)}
	\item \texttt{sort(x)}
\end{enumerate}	
\end{multicols}
\end{frame}

\begin{frame}
	Given the array \texttt{A = [2 6 9 7 ; 3 1 4 6 ; 5 3 2 7]}, explain the results of the
	following commands:

\begin{multicols}{2}
	\begin{enumerate}[(i)]	
\item \texttt{A'}
\item \texttt{A(:,[1 4])}
\item \texttt{A([2 3],[3 1])}
\item \texttt{reshape(A,2,6)}
\item \texttt{A(:)}
\item \texttt{flipud(A)}
\item \texttt{fliplr(A)}
\item \texttt{[A A(end,:)]}
\item \texttt{A(1:3,:)}
\item \texttt{[A ; A(1:2,:)]}
\item \texttt{sum(A)}
\item \texttt{sum(A')}
\item \texttt{sum(A,2)}
%\item [ [ A ; sum(A) ] [ sum(A,2) ; sum(A(:)) ] ]
\end{enumerate}	
\end{multicols}
\end{frame}

\begin{frame}
	\huge
\[
	A= \left(\begin{matrix}
	1 & 5 & 7 \\
	4 & -2 & 9\\
	3 & 1 & 4\\
	\end{matrix}\right)\;
			B = \left(\begin{matrix}
		1 & 3 & 5 \\
		1 & 0 & 3\\
		2 & 7 & -1\\
		\end{matrix}\right)
\]
\end{frame}
\begin{frame}
	\huge
	\[
	C= \left(\begin{matrix}
	1 & 2 & 2 \\
	3 & 1 & 3\\
	\end{matrix}\right)\;
	D = \left(\begin{matrix}
	1 & 5  \\
	-1 & 3\\
	2  & -2\\
	\end{matrix}\right)
	\]
\end{frame}
\begin{frame}
	\huge
The logistic function
	\[
f(x) = \frac{1}{1+e^{-x}}
	\]
\end{frame}
\begin{frame}
	\huge
	The tanh function
	\[
	f(x) = \frac{e^{x}-e^{-x}}{e^{x}+e^{-x}}
	\]
\end{frame}
\begin{frame}
	\huge
	The standardization function: \smallskip
	\[
	f(x_i) = \frac{x_i-\bar{x}}{s_x}
	\]
	
\end{frame}
\begin{frame}
	\huge
	The normalization function: \smallskip
	\[
	f(x_i) = \frac{x_i-\operatorname{min}(x)}{\operatorname{max}(x)-\operatorname{min}(x)}
	\]
	
\end{frame}

\begin{frame}[fragile]
\textbf{Statistics Functions}
\begin{framed}
\begin{verbatim}
julia> using StatsBase, StatsFuns, StreamStats
\end{verbatim}
\end{framed}	

	\begin{multicols}{3}
		\begin{enumerate}[(i)]	
			\item \texttt{mean(X)}
			\item \texttt{median(X)}
			\item \texttt{var(X)}
			\item \texttt{std(X)}
		    \item \texttt{iqr(X)}
			\item \texttt{skewness(X)}
			\item \texttt{kurtosis(X)}
			\item \texttt{mode(X) / modes(X)}
			\item \texttt{maximum(X)}
			
			\item \texttt{minimum(X)}
			\item \texttt{findmax(X)}
			\item \texttt{quantile(X)}
				\item \texttt{cov(X,Y)}
				\item \texttt{cor(X,Y)}
		%	\item \texttt{summarystats(X)}
		\end{enumerate}	
	\end{multicols}
\end{frame}

\end{document}
