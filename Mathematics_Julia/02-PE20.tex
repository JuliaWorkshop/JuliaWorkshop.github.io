
%===========================================================================%



\frametitle{Introduction to Julia : Project Euler}



n! means \[ n \times  (n − 1) \times  ... \times  3 \times  2 \times  1
 
\]

For example, \[10! = 10 \times  9 \times  ... \times  3 \times  2 \times  1 = 3628800,\]


and the sum of the digits in the number 10! is 3 + 6 + 2 + 8 + 8 + 0 + 0 = 27.
 


Find the sum of the digits in the number 100!




\end{verbatim}




%===========================================================================%



\frametitle{Introduction to Julia : Project Euler}




\begin{verbatim}


\end{verbatim}



%===========================================================================%

\end{document}
