2^15 = 32768 and the sum of its digits is 3 + 2 + 7 + 6 + 8 = 26.
 
What is the sum of the digits of the number 2^{1000}?


Sure thing! To find the sum of the digits of \(2^{1000}\), you can follow these steps using Julia:

1. Calculate \(2^{1000}\).
2. Convert the result to a string to access each digit.
3. Convert each digit back to an integer.
4. Sum all the digits.

Here's the implementation in Julia:

```julia
function sum_of_digits(base, exponent)
    # Step 1: Calculate 2^1000
    number = big(base)^big(exponent)
    
    # Step 2: Convert the number to a string
    number_str = string(number)
    
    # Step 3: Convert each character to an integer and sum them
    digit_sum = sum(c -> parse(Int, c), number_str)
    
    return digit_sum
end

# Calling the function for 2^1000
result = sum_of_digits(2, 1000)
println("The sum of the digits of 2^1000 is: ", result)
```

This should give you the sum of the digits of \(2^{1000}\). Enjoy playing around with Julia! 😄

Feel free to ask if you have any questions or if you'd like to explore more ideas.